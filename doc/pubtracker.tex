\documentclass[a4paper,10pt]{scrartcl}

%required packages
\usepackage{graphicx}					% Images and Graphics
\usepackage{hyperref}					% Hyperlinks
\usepackage[longnamesfirst]{natbib}		% Harvard Referencing System
\usepackage[utf8]{inputenc}				% UTF8 Encoding
\usepackage{eurosym}					% € Symbol
\usepackage{color}						%
\usepackage{multirow}					% 
\usepackage{longtable}					% Multi-page tables
\usepackage[printonlyused]{acronym}		% 
\usepackage[british]{babel}				% British English
\selectlanguage{british}				% indeed
\usepackage{listings}

% page layout
\textheight220mm
\frenchspacing
\parskip 8pt
\parindent 0pt

% avoid orphans and widows
\clubpenalty 10000
\widowpenalty 10000
\displaywidowpenalty 10000

\title{PubTracker\\A mobile Web Application\\{\small Continuous Assessment in Distributed and Mobile Computing}}
\author{Felix Middendorf\\{\small X00069339}}
\date{29.04.2009}

% here we go
\begin{document}
% title page
\maketitle
\begin{abstract}
	Blabla
\end{abstract}
\tableofcontents
\newpage

% finally, content

\section{Introduction}

\section{Requirement Specification}

\subsection{Features}

\subsection{Design Goals}
The following design goals are to be achieved with descending importance:
\begin{itemize}
	\item A good user experience despite the limitations of mobile clients.
	\item Can be used with a number of different mobile clients.
	\item Adheres to web standards as defined by the \ac{W3C}.
	\item Underlying data can easily be manipulated by an administrator.
 	\item Only open-source software should be required to operate the application.
\end{itemize}

\subsection{Networking}

\section{Application Design}
\subsection{Backend}

\subsection{Frontend}

\section{Installation}
\subsection{Database}
A SQLite database is set up as follows.
\begin{lstlisting}[language=bash]
cat schema.sql | sqlite3 data.db
\end{lstlisting}
Pubtracker brings along some demo user accounts and pubs in a seperate file.
\begin{lstlisting}[language=bash]
cat demodata.sql | sqlite3 data.db
\end{lstlisting}
Afterwards the chmod command needs to be used in order to grant the user who
runs php write access to these files.

\begin{appendix}

\section{Abbreviations}
\begin{acronym}
 \acro{CSS}{Casacading Style Sheets}
 \acro{HTML}{Hypertext Markup Language}
 \acro{HTTP}{Hypertext Transfer Protocol}
 \acro{HTTPS}{Hypertext Transfer Protocol Secure}
 \acro{JS}{JavaScript}
 \acro{PIN}{Personal Identification Number}
 \acro{SSL}{Secure Socket Layer}
 \acro{URL}{Uniform Resource Locator}
 \acro{W3C}{World Wide Web Consortium}
 \acro{XML}{Extensible Markup Language}
\end{acronym}
\end{appendix}

\end{document}