\documentclass[a4paper,10pt]{scrartcl}

%required packages
\usepackage{graphicx}					% Images and Graphics
\usepackage{hyperref}					% Hyperlinks
\usepackage[longnamesfirst]{natbib}		% Harvard Referencing System
\usepackage[utf8]{inputenc}				% UTF8 Encoding
\usepackage{eurosym}					% € Symbol
\usepackage{color}						%
\usepackage{multirow}					% 
\usepackage{longtable}					% Multi-page tables
\usepackage[printonlyused]{acronym}		% 
\usepackage[british]{babel}				% British English
\selectlanguage{british}				% indeed
\usepackage{listings}

% page layout
\textheight220mm
\frenchspacing
\parskip 8pt
\parindent 0pt

% avoid orphans and widows
\clubpenalty 10000
\widowpenalty 10000
\displaywidowpenalty 10000

\title{PubTracker\\A mobile Web Application\\{\small Continuous Assessment in Distributed and Mobile Computing}}
\author{Felix Middendorf\\{\small X00069339}}
\date{29.04.2009}

% here we go
\begin{document}
% title page
\maketitle
\begin{abstract}
	This paper walks through the design decisions in the building of PubTracker --
	a mobile web application which helps friends to meet spontaneously when they
	go out.
\end{abstract}
\newpage
\tableofcontents
\newpage
% finally, content

\section{Introduction}
PubTracker is a mobile web application which facilitates spontaneous meetings
of friends in the evening by allowing them to exchange their current location,
i.e.\ the pub they are currently in.

\section{Requirements}
\subsection{Targeted Users}\label{r:user}
At the current stage PubTracker is a self-hosted solution built for a closed
group of friends or acquaintances, e.g. a class in college. This implies that everyone allows the
everyone else to see his or her status updates.\\
Future versions of PubTracker may be open to the general public and include a
permission management system that allows users to manage a list of friends
which are allowed to see status updates.

\subsection{Features}
\begin{itemize}
  \item Log in using a user name and a password.
  \item Users can update their own status by selecting their current position
  from a number of stored locations.
  \item A list of pubs is available to support the decision where to go.
  \item A list of all users who are currently in a pub is available, too.
  \item A table of all active users including their current location is
  available.
  \item Status updates expire after a reasonable period of time.
\end{itemize}

\subsection{Design Goals}
The following design goals are to be achieved with descending importance:
\begin{itemize}
	\item High usability despite the limitations of mobile clients.
	\item Honouring the cost implications of mobile networking.
	\item Can be used with a variety of different (low-end) mobile clients.
	\item Adheres to the relevant web standards as defined by the \ac{W3C}.
	\item Underlying data can easily be manipulated by an administrator.
 	\item Only open-source software should be required to operate the application.
\end{itemize}

\section{Application Design}
\subsection{Standards and Best-Practices}
The site is built based on a collection of best-practices which were collected
and edited by Luca Passani \cite{passani}. It adheres to the \ac{XHTML-MP}
standard as defined by  the \ac{OMA} and recommendations by the \ac{W3C}
\cite{oma:standard, w3c:recommendations}.\\
All pages are valid \ac{XHTML-MP} and have been checked with the
\href{http://validator.w3.org/}{Markup Validation
Service}\footnote{\url{http://validator.w3.org/}} of the \ac{W3C}. The \ac{CSS}
used throughout the page was validated with the \ac{W3C}'s \href{http://jigsaw.w3.org/css-validator/}{CSS Validation
Service}\footnote{\url{http://jigsaw.w3.org/css-validator/}}.\\
PubTracker was tested with simulators of
\href{http://developer.openwave.com/dvl/tools_and_sdk/phone_simulator/}{OpenWave}%
\footnote{\url{http://developer.openwave.com/dvl/tools_and_sdk/phone_simulator/}}
and \href{http://www.opera.com/mini/demo/}{Opera
Mini}\footnote{\url{http://www.opera.com/mini/demo/}}.

\subsection{Client Recognition}
Mobile clients are recognised based on the ``X-Wap-Profile'' and the ``Accept''
headers of their \ac{HTTP} requests. If PubTracker identifies a mobile
device, the mime type ``application/vnd.wap.xhtml+xml'' is served. Although
PubTracker is targeted at mobile clients, users can visit the site using more
sophisticated devices, e.g. a netbook or a personal computer, too. In this case
they are served a \ac{XHTML} 1.0 Strict version in ``application/xhtml+xml''.

\subsection{Layout}
In order to reduce the number of \ac{HTTP} requests, the traffic and costs,
PubTracker does not make use of images at all. Instead, colors, characters and
symbols are used. For instance, each menu link is accompanied by an opening guillemet («)
which looks a bit like an arrow. Users generally associate this symbol with the
backward button, e.g.\ of a stereo device.\\
\ac{CSS} are used to improve the readability, e.g. high contrast ``zebra
tables''.\\ The application should be more or less self-explanatory. The copy used in the
application is reduced, clear and concise in order to accommodate the
constraints imposed by the little screen. Thus, little
scrolling is necessary to use the page.\\
In general, the most important information and links are displayed right at the
top of the page (with the exception of the branding headline).

\subsection{Usability}
Most links have access keys that are intuitively clear to the user. A link back
to the menu is included at the end of each page. The access key of this key is
the same everywhere: no matter where, pressing the zero key opens the menu.\\
Users do not have to enter large amounts of text using the keypads of their
mobile phones. Wherever possible text input is replaced by selectors or links.
Thus, users only have to enter text when they login. However, due to the
technique which is used to identify user session (more about this in
\ref{security}), the user can ``bookmark'' his login. Thus, he theoretically
only has to enter it once.

\subsection{CSS}
\acs{CSS} are used to improve the usability of the site and brand it.
However, only little is used to achieve the aforementioned effect. The minified
version is only about 450 bytes in size.\\
PubTracker's use of \ac{CSS} does not fully comply with Passani's best
practices \cite{passani}. This is by intention for the following reasons.
Passani advocates to use only inline \ac{CSS} in order to avoid loading and
parsing another file.\\
However, a large percentage of the PubTracker stylesheet
is used by every single page. Thus, by extracting all \ac{CSS} into a single
file it can be easily cached by the browser. PubTracker minifies the
stylesheets before sending them to the client and includes an ``Expires''
header that is far in the future. Thanks to this the impact of the failure
of buggy yet popular devices to cache hitherto retrieved \ac{CSS}, which Passani
warns of, is not considerably high. There is certainly a trade-off here.

\subsection{Security}\label{security}
The software can be operated both over \ac{HTTP} or \ac{HTTPS} depending on the
configuration of the web server. However, \ac{HTTPS} might be overkill
as the data involved can be considered non-sensitive.\\
Instead of cookies, which cause a lot problems on many devices due to poor
support, the different users are identified via a token that is part of
the \ac{URL}. When the user logs in, a unique token is randomly generated and
stored in his record in the database. The users client is now given the token
and embeds it in the \ac{URL}. Thus, the system is able to identify the user's
session by looking up the received token in the database. The token generation
is undertaken with every single login. Thus, a new login destroys older
sessions. This guarantees that an account is only used by one client at a
time.\\
All passwords are stored as salted hashes. The hash algorithm is the rather
weak md5, though \cite{ptacek}. However, the used algorithm can easily be
altered.

\subsection{Database and Administration}
By default PubTracker uses SQLite, an embedded, single file \ac{RDBMS} that
offers an \ac{SQL} interface, to store all its data. SQLite is supported by a
number of tools. Thus the data can comfortably manipulated.

\section{Networking Considerations}
PubTracker is very likely to be a low-traffic site during the day. If
used in Ireland, the activity is likely to be between 7p.m.\ and 4a.m. Most
people would use such an application on a Friday or Saturday night.\\
Up to five times an evening seems to be a reasonable estimate of the amount of
interactions per person.

Such a typical user interaction includes:
\begin{itemize}
  \item Displaying the menu (login can be omitted thanks to bookmark)
  \item Displaying the list of people
  \item Displaying a pub
\end{itemize}
According to measurements using FireBug, this has a footprint of less than 3kb.
The status update that would typically follow later adds another 2kb. Of
course, these numbers highly depend on the number of people actively using the
site -- the more people post their updates, the more \ac{XHTML-MP} is generated
and has to be transported.

Due to the fact that PubTracker does not use any other elements than
\ac{XHTML-MP} pages and only a single \ac{CSS} that is likely to be cached, only
one, at most two \ac{HTTP} requests per page are made.

In order to reduce the amount of traffic, the \ac{CSS} is minified. Moreover,
the \ac{XHTML-MP} code is rather minimalistic and stripped down to the bare
minimum necessary. Many pages are considerably smaller than 1kb in size; most
smaller than 2kb depending on the activity of the users.\\
The traffic footprint
could possibly be further reduced by compressing the data before sending it, e.g. with ``mod\_deflate'' if the Apache Web Server
is used. However, this is more of a web server configuration issue and PubTracker does
not need to be altered for this.

These considerations show that at the current stage a very basic \ac{VPS}
should be sufficient to host such a site. There is no need to worry about
requests per second and peak load times until the service is extended and
opened for public (see \ref{r:user}).

\section{Installation and Set-Up}
\subsection{Dependencies}
On Debian GNU/Linux and related distributions, e.g. Ubuntu, the following
commands install all dependencies. Root rights are necessary.
\begin{lstlisting}[language=bash]
# update the list of available packages first
apt-get update
apt-get install php5 php5-sqlite3 sqlite3
\end{lstlisting}

\subsection{Database}
PubTracker uses SQLite. Although not tested, other \acs{RDBMS} might work as
well if a \ac{PDO} driver exist. \ac{PDO}'s connection string can be configured in
``config.php". A SQLite database is set up as follows.

\begin{lstlisting}[language=bash]
cat schema.sql | sqlite3 data.db
\end{lstlisting}

Pubtracker brings along some demo user accounts and pubs in a seperate file.

\begin{lstlisting}[language=bash]
cat demodata.sql | sqlite3 data.db
\end{lstlisting}

Afterwards the chmod command needs to be used in order to grant the user who
runs php write access to these files.

\subsection{Webserver}
The ``DocumentRoot'', as it is called in the Apache Web Server, should point to
the ``www'' subdirectory. The ``open\_basedir'' option of php should be set to
PubTracker's base directory (very likely to be named ``pubtracker'').

\appendix
\clearpage
\phantomsection
\addcontentsline{toc}{section}{References}
\bibliography{references}
\bibliographystyle{plain}
\bibstyle{plain}

\section{Abbreviations}
\begin{acronym}[XHTML-MP]
 \acro{CSS}{Casacading Style Sheets}
 \acro{HTML}{Hypertext Markup Language}
 \acro{HTTP}{Hypertext Transfer Protocol}
 \acro{HTTPS}{\acs{HTTP} Secure}
 \acro{JS}{JavaScript}
 \acro{OMA}{Open Mobile Alliance}
 \acro{PDO}{PHP Data Objects}
 \acro{RDBMS}{Relational Database Management System}
 \acro{SQL}{Structured Query Language}
 \acro{URL}{Uniform Resource Locator}
 \acro{W3C}{World Wide Web Consortium}
 \acro{XML}{Extensible Markup Language}
 \acro{XHTML}{Extensible \acs{HTML}}
 \acro{XHTML-MP}{\acs{XHTML} Mobile Profile}
 \acro{VPS}{Virtual Private Server}
\end{acronym}

\section{Declaration of Authorship}
The work herein contained is the exclusive work of Felix Middendorf.

29th April 2009

\end{document}
